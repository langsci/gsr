\documentclass[a4paper]{article}
\usepackage{libertine}
\usepackage{xspace,booktabs,xcolor}
\newcommand{\stylepath}{}
\usepackage{url}
\urlstyle{same}

\usepackage{fontspec}
\newfontfamily\cn[Mapping=tex-text,Ligatures=Common,Scale=MatchUppercase]{AR PL UMing CN}
\newcommand{\zh}[1]{{\cn #1}}
\XeTeXlinebreaklocale 'zh'  
\XeTeXlinebreakskip = 0pt plus 1pt

\usepackage{langsci-gb4e}
\usepackage{langsci-lgr}
\usepackage{langsci-basic}
%set italics font for examples
\renewcommand{\exfont}{\normalsize\itshape}
\let\eachwordone=\itshape

%colour examples and example strings in blue
\newenvironment{gsrexq}{\begin{quote}\color{blue}}{\end{quote}}
\newcommand{\gsrex}[1]{{\color{blue}#1}}
\newcommand{\eagsr}{\bgroup\color{blue}\ea}
\newcommand{\zgsr}{\z\egroup}
\newcommand{\sectref}[1]{\S\ref{#1}}

% provide non-LGR abbreviation
\newcommand{\RL}{\textsc{rl}}

\date{\today}
\title{The Generic Style Rules for Linguistics: Abridged LangSci version}
% \author{Martin Haspelmath}

\begin{document}
\maketitle 

Scientists have
certain style rules for structural aspects of their research papers and
monographs, which in the past were primarily set and enforced by the
academic publishers. But in the 21st century, science is increasingly
international and research papers spread easily even without the
publishers' copy-editors and style watchers. 

This does not mean that
there is no need for style rules anymore. It makes our research and our
publication activities easier if we agree on a common set of conventions
for frequently recurring structural aspects of our writings, of the sort
that are commonly prescribed in journal style sheets (such conventions
are called \textsc{text-structure} style here). But it is inefficient if these
rules are set by individual journal editors or publishers, because
scientists usually publish in diverse venues, and being forced to apply
different style rules in different papers is an unnecessary burden on
the authors. If linguists could agree on a set of rules, then
linguistics publishers would probably be happy to adopt them sooner or
later, because they would no longer have to worry about enforcing their
house styles. 

For the specific case of formatting rules for
bibliographical references, this has already happened: In 2007, a number
of linguistics journal editors agreed on a ``Unified Style Sheet for
Linguistics'',\footnote{\url{https://linguistlist.org/pubs/journals}} and these rules for
bibliographical reference style have been widely adopted, not just for
journals, but also for linguistics books. 

Another aspect of form style
has been widely adopted: The Leipzig Glossing Rules for interlinear
morpheme-by-morpheme glosses.\footnote{\url{http://www.eva.mpg.de/lingua/resources/glossing-rules.php}}
Quite a few
journals and publishers now recommend or prescribe their use, and many
authors refer to them. The Leipzig Glossing Rules are now typically
taught in linguistics classes, and more and more linguists find it
normal that knowing them is part of their disciplinary competence. 

The following style rules for formal aspects of linguistics papers were
formulated in the same spirit. Linguistics papers have been converging
in their text-structure style over the last 20 years anyway, and while
there are still a number of things that are sometimes done differently,
none of the following rules will be particularly controversial. In most
cases, the rules reflect majority usage, and none of the rules
represents an innovation. Where they do not appear to reflect majority
usage, they are always simpler than the majority usage (e.g.~eliminating
poorly motivated exceptions to general rules). Text- structure style for
scientific papers should primarily be practical and can often leave
aside purely aesthetic considerations. 

The present style rules focus on
special conventions for linguistics-specific aspects like numbered
example sentences and the representation of expressions from other
languages, but also provide guidance for many other aspects of
text-structure style which should be uniform across a paper (or an
edited volume), and probably also across the papers of a journal. The
rules do not say anything about more specific notational conventions
that are relevant only to certain subcommunities of linguists, e.g.~for
syntactic tree representation, transcription of spoken dialogue,
optimality-theoretic tableaux, and so on. (More specific style documents
would be needed for each of these.)

Nothing is said here about typographic features such as font type, font
size, indentation and line spacing, let alone about margin and paper
size. Traditional journal style sheets often specify these features as
well, but such text-design features are aspects of typesetting, not of
text structure. There is also nothing here about (``editorial style``)
matters such as English spelling (e.g.~hyphenation), comma use, generic
pronoun use or date format, as these are issues that are not specific to
linguistics and rarely present problems in editing linguistics papers.
The present rules are also different from journal style sheets in that
they do not give instructions for submitting a paper for typesetting,
but concern the form of a paper as it should look to the reader. The
reason for this is that while submission rules will continue to depend
on diverse typesetting technologies, there is no reason why linguists
should not agree on the way certain formal aspects of their papers
should appear to the reader (i.e.~on text-structure style). 

Occasionally
the rules below make reference to some other prominent stylesheets,
especially those of the journals 
\textit{Journal of Linguistics} (Cambridge),
\textit{Language} (LSA), 
\textit{Linguistic Inquiry} (MIT Press), and the
``Stylesheet for De Gruyter Mouton journals''.\footnote{%
  \textit{Journal of Linguistics}:
    \url{http://assets.cambridge.org/LIN/LIN_ifc.pdf} \\
  \textit{Language}:
    \url{http://www.linguisticsociety.org/lsa-publications/language}\\
  \textit{Linguistic Inquiry}:
    \url{http://www.mitpressjournals.org/userimages/ContentEditor/1377619488121/LI_Style_Sheet_8.20.13.pdf}\\
  {De Gruyter Mouton}:
    \url{http://www.degruyter.com/staticfiles/pdfs/mouton_journal_stylesheet.pdf}
} 
There is no systematic comparison, but some cross-references seem
useful to make readers aware of certain salient differences between
styles. 

The Generic Style Rules may be occasionally updated in the
future. Readers are invited to send comments to Martin Haspelmath
(haspelmath@eva.mpg.de). They are published with a CC-BY licence, so
anyone is free to put them on their website.

\section{Parts of the text}\label{sec:parts}

The text of an article begins with the title, followed by the name of
the author. 

Articles are preceded by an abstract of 100--300 words. About five keywords are given. 

Articles are subdivided into sections.
Sections have a heading and are usually numbered. The number 0 never
occurs. 

More than three levels of subsections should only be used in
special circumstances. If this cannot be avoided, unnumbered subsection
headings are possible. 

The last numbered section may be followed by
several optional sections (\gsrex{Sources},
\gsrex{Acknowledgements},
\gsrex{Abbreviations},
in this order), and by one or more sections called \gsrex{Appendix} (A, B, etc.).


The last part is the alphabetic list of bibliographical references
(\gsrex{References}). For the style of references, see \sectref{sec:listofreferences} below. 

If a
(sub-)section has (sub-)subsections, there must be minimally two of
them, and they must be exhaustive. This means that all text in a chapter
must belong to some section, all text within a section must belong to
some subsection, and so on. A short introductory paragraph is allowed by
way of exception. 

Section headings do not end with a period, and have no
special capitalization (see \sectref{sec:capitalization}). 

For the parts of monographs and edited
volumes, see \sectref{sec:monoev} below.

\section{Capitalization}\label{sec:capitalization}

Sentences, proper names and titles/headings/captions start with a
capital letter, but there is no special capitalization (``title case'')
within English titles/headings neither in the article title nor in
section headings, captions or table headers.\footnote{Note that the title
of the present document has special capitalization because the Generic
Style Rules for Linguistics is a name.} Book titles in the references
do not have special capitalization either, regardless of the usage in
the original publication (but English journal titles and series titles
do, as these are treated as proper names). Thus, we have:

\begin{gsrexq}
1.1 Overview of the issues 

\qquad (NOT: Overview of the Issues) 

\noindent 
Figure 3. A schematic representation of the workflow 

\qquad (NOT: A Schematic Representation of the Workflow) 

\noindent
Anderson, Gregory. 2006. \textit{Auxiliary verb constructions}. Oxford: Oxford University Press.

\qquad (NOT: \textit{Auxiliary Verb Constructions})
\end{gsrexq}


Capitalization is used only for parts of the article (chapters, figures,
tables, appendixes) when they are numbered, e.g.

\begin{gsrexq}
  as shown in Table 5\\
  more details are given in Chapter 3\\
  this is illustrated in Figure 17\\
\end{gsrexq}


Capitalization is also used after the colon in titles, i.e.~for the
beginning of subtitles: 

\begin{gsrexq} 
  Clyne, Michael (ed.). 1991. \textit{Pluricentric languages: Different norms in different nations.} Berlin: Mouton de Gruyter.
\end{gsrexq}


\section{Italics}\label{sec:italics}

Italics are used in the following cases:
\begin{itemize}
\item For all object-language forms (letters, words, phrases, sentences) that are cited within the text or in numbered examples (see \sectref{sec:numberedexamples}). Phonetic transcriptions or phonological representations are not object-language forms.
\item For book titles, journal titles, and film titles.
\item When a technical term is referred to metalinguistically (in such contexts, English technical terms are thus treated like object-language forms), e.g.
  \begin{gsrexq}
    the term \textit{quotative} is not appropriate here
  \end{gsrexq} 
  \begin{gsrexq}
    I call this construction \textit{quotative}.
  \end{gsrexq}
\item For emphasis of a particular word that is not a technical term, e.g.
  \begin{gsrexq}
    This is possible here, but \textit{only} here.
  \end{gsrexq} 
\item Italics are not used for commonly used loanwords such as \gsrex{ad hoc}, \gsrex{façon de parler}, \gsrex{e.g.}, \gsrex{et al.}, \gsrex{Sprachbund}.
\item For emphasis within a quotation, with the indication [emphasis mine] at the end of the quotation.
\end{itemize}


\section{Small caps}\label{sec:small-caps}

Small caps are used to draw attention to an important term at its first
use or definition,\footnote{This is in line with the Language stylesheet.
The De Gruyter stylesheet requires italics for this purpose.} e.g.

\begin{gsrexq}
On this basis, the two main alignment types, namely 
\textsc{nominative-accusative}
and 
\textsc{ergative-absolutive}, 
are distinguished. 
\end{gsrexq}

Small caps are also used
for category abbreviations in interlinear glossing (see \sectref{sec:abbreviations}, \sectref{sec:numberedexamples}), and
they may be used to indicate stress or focusing in example sentences:

\eagsr
  John called Mary a Republican and then \textsc{she} insulted \textsc{him}.
\zgsr

\section{Boldface and other
highlighting}\label{sec:boldface-and-other-highlighting}

Boldface is generally not used. The only exception is to draw the
reader's attention to particular aspects of a linguistic example,
whether given within the text or as a numbered example. An illustration
is the relative pronoun \textit{dem} in (4) in \sectref{sec:numberedexamples} below. 

ALLCAPS and
underlining are not normally used for highlighting. Exceptionally,
underlining may be used to highlight a single letter in an example word,
and in other cases where other kinds of highlighting would not work.

\section{Quotation marks}\label{sec:quotation-marks}

Single quotation marks are used exclusively for linguistic meanings
(glosses, translations). Double quotation marks are used for all other
cases (citations, scare quotes):

\begin{itemize}
\item  When a passage from another work is cited in the text, e.g.\footnote{But note that blockquotes do not have quotation marks.}
    \begin{gsrexq} 
      According to Takahashi (2009: 33), ``quotatives were never used in subordinate clauses in Old Japanese''.
    \end{gsrexq}
\item When a technical term or other expression is mentioned that the author does not want to adopt,\footnote{Alternatively, italics could be used here.} e.g.
    \begin{gsrexq}
      This is sometimes called ``pseudo-conservatism'', but I will not use this term here, as it could lead to confusion.
    \end{gsrexq}
\end{itemize}

Single quotation marks are used exclusively for
linguistic meanings,\footnote{The distinction between single and double
quotation marks is not made by \textit{Language} and \textit{Journal of Linguistics}, but
is very useful and is practiced widely (e.g.~required by the \textit{Linguistic
Inquiry} and De Gruyter Mouton stylesheets).} e.g.

\begin{gsrexq}
  Latin \textit{habere} `have' is not cognate with Old English \textit{hafian} `have'. 
\end{gsrexq}
Quotes within quotes are
not treated in a special way.

Note that quotations from other languages should be translated (inline
if they are short, in a footnote if they are longer).

\section{Other punctuation matters}\label{sec:other-punctuation-matters}

The n-dash (\gsrex{--}) surrounded by spaces is used for parenthetical remarks -- as in this example -- rather than the m-dash (\gsrex{---}). The n-dash is also used for number ranges, but not surrounded by spaces
(e.g.~\gsrex{1995--1997}).

Ellipsis in a quotation is indicated by \gsrex{\ldots}. 

Angle brackets are used for specific reference to
written symbols, e.g.~\gsrex{the letter \textlangle q\textrangle}.

\section{Abbreviations}\label{sec:abbreviations}

Abbreviations of uncommon expressions should be avoided in the text.
Language names should not normally be abbreviated. 

The use of abbreviations is desirable for grammatical category labels in
interlinear morpheme-by-morpheme translations. (The Leipzig Glossing
Rules include a standard list of frequently used and widely understood
category label abbreviations.) 

When a complex term that is not widely
known is referred to frequently, it may be abbreviated (e.g. \gsrex{\textsc{doc}} for
``double-object construction''). The abbreviation should be given both
in the text when it is first used and at the end of the article in the
\textit{Abbreviations} section. 

Abbreviations of uncommon expressions are not
used in headings or captions, and they should be avoided at the
beginning of a chapter or major section.

\section{In-text citation}\label{sec:in-text-citation}

Please see the Unified Style Sheet:\\ \url{https://www.linguisticsociety.org/resource/unified-style-sheet}.

\section{Numbered examples}\label{sec:numberedexamples}

A hallmark of many linguistics articles is the use of numbered examples.
Unless they are from English (or more generally, the language of the
article), they must be glossed and translated. Glossing refers to the
use of interlinear word-by-word or morpheme-by- morpheme translations,
as described in detail in the Leipzig Glossing Rules. 

Example numbers
are enclosed in parentheses. When there are multiple examples
(``sub-examples'') under a single number, they are distinguished by the
letters a, b, etc.

\eagsr
  \eagsr She saw him. 
  \ex He saw her.
  \zgsr
\zgsr

But when a numbered example is not glossed and translated (i.e.~in
English works, when it is from English), it may be in roman (non-italic)
type. Thus, (2a-b) could alternatively be printed in roman.

Cross-references to examples use numbers in parentheses as well,\footnote{This is the most widespread practice, although the Language stylesheet omits the parentheses in cross- references.} but when a cross-reference occurs inside parentheses, the parentheses around the numbers can be omitted:

\begin{gsrexq}
  As shown in (6) and (8-11), this generalization extends to transitive constructions, but (29b) below constitutes an exception.
\end{gsrexq}

\begin{gsrexq}
  In all other environments, the stress is on the second syllable (see 15a-d).
\end{gsrexq}

When an example is from a language other than the language of the main text, it
is provided with an interlinear gloss (with word-by-word alignment) in
the second line, as well as with an idiomatic translation in the third
line, e.g.

\eagsr
\gll Storm-ur-inn  rak   bát-inn      á land.\\
     storm-\NOM-\DEF{} drove boat.\ACC-\DEF{} on land\\\jambox{\upshape (Icelandic)}
\glt     `The storm drove the boat ashore.'
\zgsr

The precise conventions for interlinear glossing are given in the
Leipzig Glossing Rules, which have become a worldwide standard. The most
important principle is that each element of the primary text corresponds
to an element in the gloss line, and boundary symbols (especially the
word-internal boundary symbol - and the clitic boundary symbol =) have
to be present both in the primary text and in the gloss. Abbreviated
category labels are set in small capitals, and the idiomatic translation
is surrounded by single quotes. A list of abbreviations is provided at
the end of the article (or at the beginning of a monograph).

Example sentences usually have normal capitalization at the beginning and normal
punctuation (usually a period) at the end. The gloss line has no
capitalization and no punctuation. The idiomatic translation again has
normal capitalization and punctuation, as seen in (3) above. When the
example is not a complete sentence, as in (4), there is no
capitalization and no punctuation.

\eagsr
\gll  das Kind, 	\textbf{dem} du 	     geholfen hast\\
      the child.\NOM{}  who.\DAT{}    you.\NOM{} helped   have\\\jambox{\upshape(German)}
\glt  `the child that you helped' 
\zgsr

When the language is not normally used as a written language, the
primary text may lack initial capitalization and normal punctuation,
e.g.

\eagsr
\upshape Hatam\\
\gll a-yai    bi-dani mem di-ngat i \\
     2\SG-get to-me   for 1SG-see \textsc{q}\\ 
\glt `Would  you give it to me so that I can see it?' (Reesink 1999: 69)
\zgsr

When multiple languages are mentioned in a single text, the name of the
language may be given to the right of the example (as in 3-4), or in the
line next to the example number, as in (5) and (6a-b).

\eagsr
\upshape Sakha\\
  \eagsr[~]{
  \gll En  bytaan buol-uoq-uŋ\\
       you slow   be-\FUT-2\SG\\}
  \glt `You will be slow.' (Baker 2012: 7)

  \ex[*]{  
  \gll En bytaan-yaq-yŋ\\ 
	  you slow-\FUT-2\SG\\}
  \glt (`You will be slow.') (Baker 2012: 7)
  \zgsr
\zgsr

Ungrammatical examples can be given a parenthesized idiomatic
translation, as in (6b). A literal translation may be given in
parentheses after the idiomatic translation, e.g.

\eagsr
\upshape Japanese\\
\gll Tsukue no     ue  ni hon  ga      aru.\\ 
     table  \GEN{} top at book \SBJ{} be\\
\glt `There is a book on the table.' (Lit. `At the top of the table is a book.')
\zgsr

The object-language text may be given in two lines, one unanalyzed
(``surface``) line, and an analyzed line, which may contain a more
abstract representation, e.g.

\eagsr
\upshape Karbi\\
\glll amatlo          la   kroikrelo\\
      amāt=lo 	      là   krōi-Cē-lò\\
      and.then=\FOC{} this agree-\NEG-\RL\\
\glt `And then, she disagreed.' (Konnerth 2014: 286)
\zgsr

Square brackets (e.g.~to indicate constituents) are never set in
italics, even when the text is in italics.

\section{Source indications}\label{sec:source-indications}

All numbered examples should have a source with the exception of trivial
sentences. These sources are standardly given directly after the
idiomatic translation, as in the following examples (see also (5-6) and
(8) above):

\eagsr
\upshape Luganda\\ 
\gll Maama  a-wa-dde             taata  ssente.\\
     Mother she.\PRS-give-\PRF{} father money\\
\glt `Mother has given father money.' (Ssekiryango 2006: 67)
\zgsr

\eagsr
\upshape Jalonke\\
\gll  I    sig-aa     xon-ee          ma.\\
      2\SG{} go-\IPFV{} stranger-\DEF{} at\\
\glt `You are going to the stranger.' [Mburee 097]
\zgsr

When the source is not a bibliographical reference, but is the name of a
text or corpus (perhaps unpublished), as in (10), the source is given in
square brackets and the article must contain a special section at the
end where more information about the sources is given. (When the source
indication is unique and quite long, it may of course alternatively be
given in a footnote, e.g.~when it is a long URL.)

\section{Tables and figures}\label{sec:tables-and-figures}

Tables and figures are numbered consecutively (\gsrex{Table 1}, 
\gsrex{Table 2}; 
\gsrex{Figure 1},
\gsrex{Figure 2},
etc.). Each table must be mentioned in the running text and
identified by its number. They appear in the text as close as possible
to the place where they are mentioned. 

Each table and each figure has a
caption. The caption precedes a table and follows a figure. If it is not
a complete sentence, it is not followed by a period.

Tables generally
have a top line and a bottom line plus a line below the column headers,
e.g.

\begin{table}[h]
\color{blue}
\centering
\label{tab:sg-pl-of-sg}
\caption{Frequency of some English nouns (BNC)}
\begin{tabular}{lr @{\qquad} lr @{\qquad} r}
\toprule
\multicolumn{2}{c}{\SG}&
		\multicolumn{2}{c}{\PL} 
				  &\% of \SG \\
\midrule     
person  & 24671 & persons  & 4034 & 86\% \\
house   & 49295 & houses   & 9840 & 83\% \\
hare    & 488   & hares    & 136  & 78\% \\
bear    & 1182  & bears    & 611  & 65\% \\
feather & 48 7  & feathers & 810  & 38\% \\
\bottomrule
\end{tabular}
\end{table}

Footnotes should not be used in Tables. There relevant information
should be either given in the running text or in the caption.

\section{Cross-references in the text}\label{sec:cross-references-in-the-text}

Cross-references to chapters, tables, figures or footnotes use the
capitalized names for these items (e.g.
\gsrex{Chapter 4},
\gsrex{Section 5},
\gsrex{Figure 3},
\gsrex{Table 2},
\gsrex{Footnote 17}). 
Abbreviations like 
``Fig. 3'', 
``Ch. 4'', or 
``n. 17''
are not used.

Cross-references to sections can optionally use the \gsrex{§}
character (e.g. \gsrex{§2.3}).

\section{Footnotes}\label{sec:footnotes}

Superscript footnote numbers follow punctuation, though exceptionally it
may follow an individual word.

Footnote numbers start with \gsrex{1}. The
acknowledgements are printed as a separate section following the body of
the text, not as a footnote. Likewise, abbreviations and other
notational conventions are given in a separate section (following the
acknowledgements, see \sectref{sec:parts} above).

Numbered examples in footnotes have the
numbers \gsrex{(i)}, 
\gsrex{(ii)}, etc. If there are sub- examples, they have the
numbers \gsrex{(i.a)}, 
\gsrex{(i.b)}, etc.

\section{Non-Latin scripts}\label{sec:non-latin-scripts}

All forms in languages that are not normally written with the Latin
alphabet (such as Japanese or Armenian) should (additionally) be given
in transcription or transliteration.

When the article is entirely about
a particular language, the original script should not be omitted, at
least in numbered examples. 

Non-Latin forms need not be printed in
italics.

\section{List of references}\label{sec:listofreferences}

The list of references at the end of an article has the heading
\gsrex{References} (or \gsrex{Bibliography} at the end of a monograph). The entries are
listed alphabetically.

\subsection{General points}\label{sec:general-points}

For the formatting, the Generic Style Rules follow the 2007 Unified
Style Sheet for Linguistics in almost all respects. 

It should be noted especially that

\begin{itemize}
\item  the names of authors and editors should be given in their full form as in the publication, without truncation of given names (but note that some authors habitually use initials only, e.g.~\gsrex{J.~K.~Rowling} and \gsrex{R.~M.~W.~Dixon}; these count as full)
\item page numbers are obligatory
\item journal titles are not abbreviated
\item main title and subtitle are separated by a colon, not by a period.
\end{itemize}

\subsection{General formatting rules}\label{sec:general-formatting-rules}

Additional nonstandard parts may follow the reference in parentheses.

\subsection{Optional parts}\label{sec:optional-parts}

The book title may be followed by series information (series title plus
series number), given in parentheses: 
\begin{gsrexq}
Lahiri, Aditi (ed.). 2000. \textit{Analogy, leveling, markedness: Principles of change in phonology and morphology} (Trends in Linguistics 127). Berlin: Mouton de Gruyter.
\end{gsrexq}

Series titles have special capitalization, like journal titles (see \sectref{sec:capitalization}). 

\subsection{Author surnames and given names}\label{sec:author-surnames-and-given-names}

Surnames with internal complexity are never treated in a special way.
\footnote{This 
  is a simplification over the 2007 Unified Style Sheet, which treats ``names with von, van, de, etc.'' in a special way.
}
Thus, Dutch or German surnames that begin with van or von (e.g.~van
Riemsdijk) or French and Dutch surnames that begin with with de (e.g.~de
Groot) are treated just like Belgian surnames (e.g.~De Schutter) and
Italian surnames (e.g.~Da Milano) and are alphabetized under the first
part, even though they begin with a lower-case letter. Thus, the
following names are sorted alphabetically (i.e.~mechanically) as
indicated. 

\begin{gsrexq}
Da Milano, Federica \\
de Groot, Casper\\
De Schutter, Georges \\
de Saussure, Ferdinand\\
{\dots}\\
van der Auwera, Johan\\
Van Langendonck, Willy\\
van Riemsdijk, Henk\\
von Humboldt, Wilhelm\\
\end{gsrexq}

When they occur in the prose text, they are not
treated in a special way either, i.e.~they have lower case unless they
occur at the begining of a sentence (this is in line with the French and
German practice,\footnote{With
  classical authors such as de Saussure and von Humboldt, the first part of the name can be (and is often) omitted. But this is not possible with modern names (e.g.~von Heusinger, never *Heusinger).
}
but in contrast to the Dutch practice), e.g.
\begin{gsrexq}
as has been claimed by van Riemsdijk \& Williams (1981)
\end{gsrexq}
Chinese and Korean names may
be treated in a special way: As the surnames are often not very
distinctive, the full name may be given in the in-text citation,
e.g.

\begin{gsrexq}
the neutral negation \textit{bù} is compatible with stative and activity
verbs (cf.~Teng Shou-hsin 1973; Hsieh Miao-Ling 2001; Lin Jo-wang 2003)
\end{gsrexq}

\subsection{Internet publications}\label{sec:internet-publications}

Regular publications that are available online are not treated in a
special way, as this applies to more and more publications anyway. 

When citing a web resource that is not a regular scientific publication, this
should be treated like a book, to the extent that this is possible,
e.g.

\begin{gsrexq}
Native Languages of the Americas. 1998--2014. \textit{Vocabulary in Native American languages: Salish words.} (\url{http://www.native-languages.org/salish_words.htm}) (Accessed 2014-12-02.)
\end{gsrexq}

Internet publications should list the DOI if it is available. In these cases, the URL is not listed. 

\subsection{Miscellaneous}\label{sec:miscellaneous}

Books may include a volume number, separated from the book title by a
comma: 

\begin{gsrexq}
Rissanen, Matti. 1999. Syntax. In Lass, Roger (ed.), \textit{Cambridge history of the English language}, vol.~3, 187--331. 
Cambridge: Cambridge University Press. 
\end{gsrexq}

And there may be information about the edition,
following the book title: 

\begin{gsrexq}
Croft, William. 2003. \textit{Typology and universals}.
2nd edn. Cambridge: Cambridge University Press.
\end{gsrexq}

If a publisher is
associated with several cities, only the first one needs to be given,
e.g.~\gsrex{Berlin: De Gruyter Mouton}, or \gsrex{Amsterdam: Benjamins}.

Other
nonstandard types of information may follow the standard parts in
parentheses, e.g.~

\begin{gsrexq}
Mayerthaler, Willi. 1988. \textit{Morphological naturalness}.
Ann Arbor: Karoma. (Translation of Mayerthaler 1981.)
\end{gsrexq}

Titles of works
written in a language that readers cannot be expected to know may be
accompanied by a translation, given in brackets: 


\begin{gsrexq}
Haga, Yasushi. 1998.
\textit{Nihongo no shakai shinri} [Social psychology in the Japanese
language]. Tokyo: Ningen no Kagaku Sha.
\end{gsrexq}


\begin{gsrexq}
Li, Rulong. 1999. Minnan fangyan de daici
[Demonstrative and personal pronouns in Southern Min]. 
In Li, Rulong \& Chang, Song-Hing (eds.), 
\textit{Daici} [\textit{Demonstrative and personal pronouns}], 263--287. 
Guangzhou: Ji'nan University Press.
\end{gsrexq}


If the title is not only in a different language, but also in a
different script, it may be given in the original script, in addition to
the transliteration (following it in parentheses). Likewise, the name of
the author may be given in the original script, as follows:


\begin{gsrexq}
Plungian, Vladimir A. (Плунгян, Владимир А.) 2000. 
\textit{Obščaja morfologija: Vvedenie v problematiku}
(Общая морфология: Введение в проблематику) 
[General morphology: Introduction to the issues]. 
Moskva: URSS. 
\end{gsrexq}


\begin{gsrexq}
Chen, Shu-chuan (\zh{陳淑娟}). 2013. Taibei Shezi fangyan de yuyin bianyi yu bianhua
(\zh{台北社子方 言的語音變異與變化}) 
[The sound variation and change of Shezi dialect in Taipei city]. 
\textit{Language and Linguistics} 14(2). 371--408.
\end{gsrexq}


The place of publication is a city. Extra information about larger
bodies (countries (Paris, France), states (Los Angeles, CA), provinces
(Vancouver, BC)) is not given.

\section{Rules for monographs and edited volumes}\label{{sec:monoev}}

\subsection{Parts of books}\label{sec:parts-of-books}

Books (as well as dissertations and other theses) consist of the
following parts (with optional parts in parentheses): title page,
colophon page, (dedication page,) contents, (acknowledgements or
preface, abbreviations or notational conventions,) chapter 1, chapter 2,
etc., appendix A, appendix B, etc., bibliography, name index, (language
index,), subject index. 

Books may also group the chapters into parts
(Part I, Part II, etc). A new part does not start a new chapter
numbering. Parts mainly serve to provide orientation in the table of
contents.

\subsection{Monographs vs.~edited volumes}\label{sec:monoev}

Chapters of edited books are preceded by an abstract, like journal
articles, but chapters of monographs are not accompanied by an abstract.

Edited books are treated like a collection of journal articles,
i.e.~each article has its own list of references and abbreviations, so
that the articles can be read and understood independently.

Chapters in
edited volumes are numbered like chapters in monographs, but the chapter
number is not contained in the section number, i.e.~Section 2 of Chapter
5 is §2, not §5.2.

\subsection{Table of contents}\label{sec:table-of-contents}

The table of contents (called \gsrex{Contents}) lists the chapters, chapter
sections and subsections (indented and preceded by their numbers) and
gives the corresponding page numbers.

\subsection{Cross-references}\label{sec:cross-references}

While articles (including articles in edited volumes) refer only to
sections within the same article, monographs may refer to chapters and
sections within the same book, and to sections within the same chapter.
Note that when referring to parts of a book, \gsrex{§2.3} means §3 of Chapter 2.

\subsection{Numbering tables and figures}\label{sec:numbering-tables-and-figures}

In monographs, the numbers of tables and figures are preceded by the
chapter number. Thus, the second table in Chapter 3 is \gsrex{Table 3.2}.
% (However, examples normally start with \gsrex{(1)} in each new chapter.)

This rule does not apply to chapters in edited volumes, as the chapter
numbers are not salient here.

\subsection{Bibliographical references}\label{sec:bibliographical-references}
\enlargethispage{1\baselineskip}
When a self-standing chapter in an edited book contains a reference to
another chapter in the same book, the referred-to chapter is listed in
the references in the normal way, as if it were published in a different
place. However, the in-text citation may contain the additional comment
\gsrex{[in this volume]} in brackets, e.g.

\begin{gsrexq}
As explained by Li \& Kim (2015 [this volume]), it is often useful to\ldots{} 
\end{gsrexq}
\end{document}
